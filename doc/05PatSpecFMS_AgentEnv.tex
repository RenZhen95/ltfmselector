Drawing some inspiration from the described example, DQL is applied in this work to select an optimal feature subset of extracted gait features, coupled along with a corresponding prediction (supervised learning) model to automatically assess gait poststroke. Earlier works by \cite{lee2020,lee2021} have applied DQL to develop a clinical decision support system (CDSS) that automatically assesses a patient's ability in performing functional exercises, while delivering patient-specific relevant features for each corresponding task. In contrast, the method developed here is $(i)$ applied to the context of gait-assessment poststroke and $(ii)$ extended to include model selection. Model selection in this work will cover both the selection of a \emph{learning algorithm} and the subsequent \emph{hyperparameter tuning}. The issue of model selection is often described as more an art than a strict science \cite{raschka2020}, where there exist no single ``best'' optimization algorithm across all problem spaces or datasets. Consequently, the selection of a model should be guided by the characteristics of the dataset. The learning algorithm and its corresponding optimal hyperparameters will be referred to as the \emph{prediction model} (PM) in the remainder of this paper.
