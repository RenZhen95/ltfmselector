Poststroke gait rehabilitation requires a personalized therapy, usually designed by an interdisciplinary medical team via time-consuming assessments \cite{raab2020,liaw2025}. An automated gait assessment tool based on gait measurements and interdisciplinary knowledge could allow for faster poststroke evaluation, while providing relevant feedback via objective analysis of a patient’s status. One major challenge of using gait data for this purpose is its high dimensionality, which is usually met by carrying out feature selection on a fixed feature set. Owing to the individual uniqueness of each patient in terms of physical and functional statuses \cite{lee2020}, we present a dynamic feature and model selection approach using reinforcement learning (RL).

\cite{lee2020} has for instance shown that when performing the ``Bring a Hand to Mouth''-exercise during stroke rehabilitation, different stroke patients compensate for the affected motion in different ways. Beyond inter-patient variability, the relevant biomarkers have also been shown to evolve alongside disease severity. \cite{pistacchi2017} showed for instance how reduced step lengths appeared to be a specific feature of Parkinson's disease in its early stages. As the disease progresses to its moderate stage, gait asymmetry, double-limb support, and increased cadence becomes more characteristic, followed by freezing of gait and reduced balance in its advanced stages. Notably, research \cite{huang2016,biase2020} have highlighted the necessity of adapting the analyzed gait parameters to evolve in tandem with the disease's condition.

In this work, we apply the Deep $Q$-Learning (DQL) algorithm by \cite{mnih2015} to select an optimal set of salient gait features, coupled along with a corresponding prediction model to automatically assess gait poststroke, based on the interdisciplinary knowledge of a medical board. This dynamic approach allows for selecting patient-specific key features, which should be more beneficial over classically selecting a fixed subset of informative features \cite{lee2020,lee2021}. It is after all the therapist's goal to design a \emph{personalized} therapy plan. Moreover, simply presenting a multitude of variables can easily overwhelm a therapist and hinder one from obtaining useful insights \cite{lee2021}. This could in turn aid a clinician in saving precious time \cite{lee2021}, especially in light of the current shortage of medical staff \cite{healthcareburden}. 
