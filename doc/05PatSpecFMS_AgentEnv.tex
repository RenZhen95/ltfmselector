Inspired by the application of RL in controlling mechanical systems (i.e. balancing an inversed pendulum), the DQL algorithm is applied in this work to select an optimal feature subset of the extracted gait features, coupled along with a corresponding prediction (supervised learning) model to automatically assess gait poststroke. Earlier works by \cite{lee2020,lee2021} have applied DQL to develop a CDSS that automatically assesses a patient's ability in performing functional exercises, while delivering patient-specific relevant features for each corresponding task. In contrast, the method developed here is $(i)$ applied to the context of gait-assessment poststroke and $(ii)$ extended to include model selection. Model selection in this work will cover both the selection of a \emph{learning algorithm} and the subsequent \emph{hyperparameter tuning}. The issue of model selection is often described as more an art than a strict science \cite{raschka2020}, a notion underpinned by the ``No Free Lunch'' theorem by \cite{nofreelunch}, which proves that no single optimization algorithm can outperform all others across all possible problem spaces or datasets. Consequently, the selection of a model should be guided by the characteristics of the dataset, instead of a reliance on a universal ``best'' model. The learning algorithm and its corresponding optimal hyperparameters will be referred to as the \emph{prediction model} (PM) in the remainder of this paper.
