Designing personalized therapy for poststroke gait rehabilitation often involves the effort of an interdisciplinary medical team and tedious assessments. An automated gait assessment tool based on gait measurements and interdisciplinary knowledge could help experts with faster gait assessments, while providing objective feedback. The high dimensionality of gait data however, makes it challenging for subsequent analyses. Inspired by the application of Deep Q-Learning in solving multi-step procedures, this study presents a method for dynamic feature and model selection. The search space is formulated as a partially observable Markov Decision Process, where the agent iteratively explores the 680 extracted gait features and various prediction models, to learn optimal patient-specific combinations of feature subsets and prediction models. The model was developed using a dataset of 904 stride pairs from 100 hemiparetic stroke patients. Each patient was evaluated by an interdisciplinary medical board using the Stroke Mobility Score, a multiple-cue clinical observational score comprised of six functional subscores. The agent was trained to approximate optimal decision-making, receiving rewards for accurate predictions and efficient feature selection. Results demonstrated excellent predictive performance, achieving a coefficient of determination ($R^2$) of 0.83 on a held-out test dataset. More importantly, the tool identifies patient-specific key features, that could help highlight specific therapeutic targets for designing personalized poststroke therapy.

%% ORIGINAL
% Designing personalized therapy for poststroke gait rehabilitation often involves the effort of an interdisciplinary medical team and tedious assessments. An automated gait-assessment tool based on gait measurements and interdisciplinary knowledge could help experts with faster gait assessments, while providing objective feedback. However, developing such a tool based on gait data can be challenging due to the high dimensionality of the training datasets typically derived from gait measurements.
%
% While a common approach involves carrying out feature selection on a fixed feature set, this work presents a dynamic feature and model selection approach using reinforcement learning. Hereby, the task of selecting the optimal feature set and corresponding prediction model to map gait data to expert gait-assessment is formulated as a partially observable Markov Decision Process (POMDP), where an agent learns by autonomously exploring different options for each available sample iteratively.
%
% This approach adds a patient-specific component to the gait assessment tool, which could assist clinicians in tailoring personalized therapy.
%
% To achieve this, a dataset is first obtained from 100 hemiparetic stroke-patients which received a clinical examination and a full-body instrumented gait-analysis. An interdisciplinary board of medical experts assigned each patient a Stroke Mobility Score, a multiple-cue clinical observational score comprised of six sub-scores, each pertaining to a functional criterion of gait.
%
% From the measurements, 904 measured stride pairs of 100 patients were obtained, 690 gait features extracted, and the dataset split 70/30 for training and testing. As a preprocessing step, expert knowledge was used to trim the features accordingly, followed by filtering out statistically non-discriminatory features. Within the setting of a POMDP, the agent is allowed to either query a feature or a prediction model, or make a prediction based on the selected feature set and prediction model.
%
% The agent is then rewarded accordingly for each action, before transitioning onto a next state where the process is repeated iteratively. Over the course of many iterations, the agent eventually learns to select an optimal set of actions, given a set of features of a stride pair measurement.
%
% The agent is trained using a Deep Q-Learning algorithm that approximates the Bellman equation by training a deep neural network iteratively on a batch of randomly chosen transitions. The trained agent tested on the test dataset yielded excellent predictive performance, showing a coefficient of determination of 0.85, while delivering patient-specific key features. The delivered patient-specific key features could help clinicians focus on key therapeutic targets, specifically tailored to a patient's needs.
