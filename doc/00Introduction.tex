Poststroke gait rehabilitation requires a personalized therapy, usually designed by an interdisciplinary medical team via time-consuming assessments \cite{raab2020,liaw2025}. An automated gait assessment tool based on gait measurements and interdisciplinary knowledge could allow for faster poststroke evaluation, while providing relevant feedback via objective analysis of a patient’s status. One major challenge of using gait data for this purpose is its high dimensionality, which is usually met by carrying out feature selection on a fixed feature set. Owing to the individual uniqueness of each patient in terms of physical and functional statuses \cite{lee2020}, we present a dynamic feature and model selection approach using reinforcement learning (RL).

\cite{lee2020} has for instance shown that when performing the ``Bring a Hand to Mouth''-exercise during stroke rehabilitation, different stroke patients compensate for the affected motion in different ways. Other than inter-patient variability, the relevant biomarkers of a patient have also been shown to change with the severity of one's condition. \cite{pistacchi2017} showed for instance how reduced step lengths appeared to be a specific feature of Parkinson's disease in its early stages, and as the disease progresses to its moderate stage, gait asymmetry, double-limb support, and increased cadence becomes more characteristic, followed by freezing of gait and reduced balance in its advanced stages. Notably, research \cite{huang2016,biase2020} have highlighted the necessity of adapting the analyzed gait parameters to evolve in tandem with the disease's condition.

Therefore, it is hypothesized that a CDSS which dynamically selects salient features per individual patient should be more beneficial over classically selecting a fixed subset of informative features \cite{lee2020,lee2021}. It is after all the therapist's goal to design a therapy plan, \emph{tailored to a specific patient}. Moreover, simply presenting a multitude of variables can easily overwhelm a therapist and hinder one from obtaining useful insights \cite{lee2021}. A CDSS that can automatically identify a subset of patient-specific relevant features should thus greatly help a therapist save precious time \cite{lee2021}, especially in light of the current shortage of medical staff \cite{healthcareburden}.

Inspired by the application of RL in controlling mechanical systems (i.e. balancing an inversed pendulum), the Deep $Q$-Learning (DQL) algorithm is applied in this work to select an optimal feature subset of the extracted gait features, coupled along with a corresponding prediction (supervised learning) model to automatically assess gait poststroke. Earlier works by \cite{lee2020,lee2021} have applied DQL to develop a CDSS that automatically assesses a patient's ability in performing functional exercises, while delivering patient-specific relevant features for each corresponding task. In contrast, the method developed here is $(i)$ applied to the context of gait-assessment poststroke and $(ii)$ extended to include model selection. Model selection in this work will cover both the selection of a \emph{learning algorithm} and the subsequent \emph{hyperparameter tuning}.

The issue of model selection is often described as more an art than a strict science \cite{raschka2020}, a notion underpinned by the ``No Free Lunch'' theorem by \cite{nofreelunch}, which proves that no single optimization algorithm can outperform all others across all possible problem spaces or datasets. Consequently, the selection of a model should be guided by the characteristics of the dataset, instead of a reliance on a universal ``best'' model. The common practice among practitioners is to employ hold out or/and cross validation techniques to evaluate and select the optimal model hyperparameters and learning algorithm \cite{nestedcv,raschka2020}. In this work, the ability of model-free RL to learn an optimal policy purely, based on the saved experiences of the agent-environment interaction is leveraged to dynamically select
\begin{enumerate}
  \item a subset of features, alongside
  \item a learning algorithm and its corresponding optimal hyperparameters,
\end{enumerate}
based on each individal patient's extracted gait features. The learning algorithm and its corresponding optimal hyperparameters will be referred to as the \emph{prediction model} (PM) in the remainder of this paper.